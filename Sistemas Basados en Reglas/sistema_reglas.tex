\documentclass[12pt]{article}
\usepackage[utf8]{inputenc}
\usepackage[spanish]{babel}
\usepackage{graphicx}
\usepackage{listings}
\usepackage{xcolor}
\usepackage{geometry}

% Configuración de márgenes
\geometry{a4paper, margin=2.5cm}

% Configuración para el código Python
\definecolor{codegreen}{rgb}{0,0.6,0}
\definecolor{codegray}{rgb}{0.5,0.5,0.5}
\definecolor{codepurple}{rgb}{0.58,0,0.82}
\definecolor{backcolour}{rgb}{0.95,0.95,0.92}

\lstdefinestyle{pythonstyle}{
    backgroundcolor=\color{backcolour},   
    commentstyle=\color{codegreen},
    keywordstyle=\color{magenta},
    numberstyle=\tiny\color{codegray},
    stringstyle=\color{codepurple},
    basicstyle=\ttfamily\small,
    breakatwhitespace=false,         
    breaklines=true,                 
    captionpos=b,                    
    keepspaces=true,                 
    numbers=left,                    
    numbersep=5pt,                  
    showspaces=false,                
    showstringspaces=false,
    showtabs=false,                  
    tabsize=2,
    language=Python,
    frame=single
}

\lstset{style=pythonstyle}

\title{Sistema Basado en Reglas para Diagnóstico de Vehículos}
\author{Rubén Jiménez}
\date{Octubre 2025}

\begin{document}

\maketitle
\tableofcontents
\newpage

\section{Introducción}

Los sistemas basados en reglas son un tipo de sistema experto que utiliza un conjunto de reglas condicionales (IF-THEN) para tomar decisiones o realizar diagnósticos. En este documento se presenta un sistema de diagnóstico de problemas en vehículos implementado en Python.

\subsection{Objetivos}
\begin{itemize}
    \item Diseñar un sistema basado en reglas para diagnóstico automotriz
    \item Implementar el sistema en Python utilizando estructuras condicionales
    \item Proporcionar diagnósticos precisos basados en síntomas reportados
\end{itemize}

\section{Enunciado del Ejercicio}

\textbf{ACTIVIDAD 4.- Diseña tu propio Sistema basado en reglas y prográmalo en Python.}

\begin{enumerate}
    \item Escoge o diseña un sistema basado en reglas
    \item Prográmalo en Python de tal forma que la respuesta sea la esperada
\end{enumerate}

\subsection{Sistema Seleccionado}
Se ha diseñado un sistema de diagnóstico de problemas en vehículos que, mediante una serie de preguntas al usuario, identifica posibles fallas y proporciona recomendaciones de reparación.

\section{Diseño del Sistema}

\subsection{Reglas del Sistema}

El sistema implementa las siguientes reglas principales:

\begin{enumerate}
    \item \textbf{SI} el vehículo enciende \textbf{Y} hace ruidos \textbf{ENTONCES} diagnosticar según ubicación del ruido
    \begin{itemize}
        \item Motor: Problema en el motor (correa, aceite, distribución)
        \item Frenos: Problema en frenos (pastillas, discos) - URGENTE
        \item Suspensión: Problema en suspensión (amortiguadores)
    \end{itemize}
    
    \item \textbf{SI} el vehículo enciende \textbf{Y} no hace ruidos \textbf{ENTONCES} evaluar rendimiento
    \begin{itemize}
        \item Alto consumo: Filtros, inyectores, neumáticos
        \item Pérdida de potencia: Bujías, sensores, filtros
        \item Humo: Aceite, pistones, compresión
    \end{itemize}
    
    \item \textbf{SI} el vehículo no enciende \textbf{Y} hace sonido \textbf{ENTONCES} problema eléctrico o combustible
    \begin{itemize}
        \item Sonido "clic": Batería descargada
        \item Motor intenta arrancar: Problema de combustible o chispa
    \end{itemize}
    
    \item \textbf{SI} el vehículo no enciende \textbf{Y} no hace sonido \textbf{ENTONCES} batería o sistema eléctrico
    \begin{itemize}
        \item Luces del tablero SI encienden: Motor de arranque dañado
        \item Luces del tablero NO encienden: Batería completamente descargada
    \end{itemize}
\end{enumerate}

\subsection{Diagrama de Flujo Textual}

\begin{verbatim}
¿El vehículo ENCIENDE?
│
├─ SÍ → ¿Hace RUIDOS extraños?
│       ├─ SÍ → ¿Dónde? (motor/frenos/suspensión)
│       │       ├─ Motor → Problema en motor
│       │       ├─ Frenos → Problema en frenos (URGENTE)
│       │       └─ Suspensión → Problema en suspensión
│       │
│       └─ NO → ¿Problemas de RENDIMIENTO?
│               ├─ SÍ → ¿Qué problema? (consumo/potencia/humo)
│               └─ NO → Vehículo en buen estado
│
└─ NO → ¿Hace SONIDO al girar llave?
        ├─ SÍ → ¿Qué tipo? (clic/arranque)
        │       ├─ Clic → Batería descargada
        │       └─ Arranque → Problema combustible/chispa
        │
        └─ NO → ¿Luces del tablero encienden?
                ├─ SÍ → Motor de arranque dañado
                └─ NO → Batería muerta/desconectada
\end{verbatim}

\newpage

\section{Implementación en Python}

\subsection{Código Completo}

\begin{lstlisting}[caption={Sistema de Diagnóstico de Vehículos}]
# Sistema basado en reglas para diagnostico de vehiculos

print("=== SISTEMA DE DIAGNOSTICO DE VEHICULOS ===\n")
print("Bienvenido al sistema de diagnostico automotriz")
print("Responda las preguntas para identificar el problema\n")

# Primera pregunta: El vehiculo enciende?
print("El vehiculo enciende? (si/no)")
enciende = input().lower()

if enciende == "si":
    # Segunda pregunta: Hace ruidos extraños?
    print("El vehiculo hace ruidos extraños? (si/no)")
    ruidos = input().lower()
    
    if ruidos == "si":
        # Tercera pregunta: Donde se escucha el ruido?
        print("Donde se escucha el ruido? (motor/frenos/suspension)")
        ubicacion = input().lower()
        
        if ubicacion == "motor":
            print("\nDIAGNOSTICO: Problema en el MOTOR")
            print("   - Posibles causas: correa desgastada, falta de aceite")
            print("   - Recomendacion: Revisar nivel de aceite y llevar a mecanico")
        elif ubicacion == "frenos":
            print("\nDIAGNOSTICO: Problema en los FRENOS")
            print("   - Posibles causas: pastillas gastadas, discos desgastados")
            print("   - Recomendacion: URGENTE! Revisar sistema de frenos inmediatamente")
        elif ubicacion == "suspension":
            print("\nDIAGNOSTICO: Problema en la SUSPENSION")
            print("   - Posibles causas: amortiguadores dañados, barra estabilizadora")
            print("   - Recomendacion: Revisar amortiguadores y tren delantero")
        else:
            print("\nDIAGNOSTICO: Revisar otras areas del vehiculo")
    
    else:
        # No hace ruidos
        print("Tiene problemas de rendimiento? (si/no)")
        rendimiento = input().lower()
        
        if rendimiento == "si":
            print("Que problema presenta? (consumo/potencia/humo)")
            problema = input().lower()
            
            if problema == "consumo":
                print("\nDIAGNOSTICO: ALTO CONSUMO de combustible")
                print("   - Posibles causas: filtro de aire sucio, inyectores, presion de neumaticos")
                print("   - Recomendacion: Revisar filtros y sistema de inyeccion")
            elif problema == "potencia":
                print("\nDIAGNOSTICO: PERDIDA DE POTENCIA")
                print("   - Posibles causas: filtros sucios, bujias, sensor MAF")
                print("   - Recomendacion: Hacer mantenimiento general y diagnostico electronico")
            elif problema == "humo":
                print("\nDIAGNOSTICO: Emision de HUMO anormal")
                print("   - Posibles causas: aceite quemandose, problemas en pistones")
                print("   - Recomendacion: Revisar nivel de aceite y compresion del motor")
            else:
                print("\nDIAGNOSTICO: Requiere revision detallada")
        else:
            print("\nDIAGNOSTICO: El vehiculo esta en BUEN ESTADO")
            print("   - Recomendacion: Continuar con mantenimiento preventivo regular")

else:
    # El vehiculo NO enciende
    print("Hace algun sonido al girar la llave? (si/no)")
    sonido = input().lower()
    
    if sonido == "si":
        print("Que tipo de sonido hace? (clic/arranque/nada)")
        tipo_sonido = input().lower()
        
        if tipo_sonido == "clic":
            print("\nDIAGNOSTICO: Problema de BATERIA")
            print("   - Posibles causas: bateria descargada o dañada")
            print("   - Recomendacion: Cargar o reemplazar la bateria")
        elif tipo_sonido == "arranque":
            print("\nDIAGNOSTICO: Problema de COMBUSTIBLE o CHISPA")
            print("   - Posibles causas: falta de gasolina, bomba de combustible, bujias")
            print("   - Recomendacion: Verificar nivel de combustible y sistema de encendido")
        else:
            print("\nDIAGNOSTICO: Problema ELECTRICO severo")
            print("   - Recomendacion: Revisar fusibles y sistema electrico completo")
    else:
        print("Las luces del tablero encienden? (si/no)")
        luces = input().lower()
        
        if luces == "si":
            print("\nDIAGNOSTICO: Problema en el MOTOR DE ARRANQUE")
            print("   - Posibles causas: motor de arranque dañado, relay defectuoso")
            print("   - Recomendacion: Revisar motor de arranque y conexiones")
        else:
            print("\nDIAGNOSTICO: BATERIA COMPLETAMENTE DESCARGADA o desconectada")
            print("   - Posibles causas: bateria muerta, terminales flojos/corroidos")
            print("   - Recomendacion: Revisar conexiones de bateria y estado de carga")

print("\n" + "="*50)
print("Gracias por usar el sistema de diagnostico!")
print("Recuerde: Este es un diagnostico preliminar.")
print("Siempre consulte con un mecanico profesional.")
print("="*50)
\end{lstlisting}

\newpage

\section{Ejemplos de Ejecución}

\subsection{Caso 1: Batería Descargada}

Entrada del usuario:
\begin{verbatim}
El vehiculo enciende? no
Hace algun sonido al girar la llave? si
Que tipo de sonido hace? clic
\end{verbatim}

Salida del sistema:
\begin{verbatim}
DIAGNOSTICO: Problema de BATERIA
   - Posibles causas: bateria descargada o dañada
   - Recomendacion: Cargar o reemplazar la bateria
\end{verbatim}

\subsection{Caso 2: Problema en Frenos}

Entrada del usuario:
\begin{verbatim}
El vehiculo enciende? si
El vehiculo hace ruidos extraños? si
Donde se escucha el ruido? frenos
\end{verbatim}

Salida del sistema:
\begin{verbatim}
DIAGNOSTICO: Problema en los FRENOS
   - Posibles causas: pastillas gastadas, discos desgastados
   - Recomendacion: URGENTE! Revisar sistema de frenos inmediatamente
\end{verbatim}

\subsection{Caso 3: Alto Consumo de Combustible}

Entrada del usuario:
\begin{verbatim}
El vehiculo enciende? si
El vehiculo hace ruidos extraños? no
Tiene problemas de rendimiento? si
Que problema presenta? consumo
\end{verbatim}

Salida del sistema:
\begin{verbatim}
DIAGNOSTICO: ALTO CONSUMO de combustible
   - Posibles causas: filtro de aire sucio, inyectores, presion de neumaticos
   - Recomendacion: Revisar filtros y sistema de inyeccion
\end{verbatim}

\section{Análisis del Sistema}

\subsection{Ventajas}
\begin{itemize}
    \item \textbf{Facilidad de uso:} El sistema guía al usuario mediante preguntas simples
    \item \textbf{Diagnóstico estructurado:} Sigue un árbol de decisión lógico
    \item \textbf{Recomendaciones claras:} Proporciona acciones específicas a tomar
    \item \textbf{Escalabilidad:} Fácil de expandir con más reglas y diagnósticos
\end{itemize}


\section{Conclusiones}

El sistema de diagnóstico implementado demuestra la efectividad de los sistemas basados en reglas para:

\begin{itemize}
    \item Realizar diagnósticos sistemáticos mediante preguntas guiadas
    \item Proporcionar recomendaciones específicas según los síntomas
    \item Facilitar la identificación de problemas comunes en vehículos
    \item Aplicar lógica condicional de manera estructurada
\end{itemize}

El sistema puede expandirse agregando más reglas y casos específicos para mejorar la precisión del diagnóstico. Además, podría integrarse con bases de datos de problemas conocidos o sistemas de aprendizaje automático para mejorar sus capacidades.


\section{Referencias}

\begin{itemize}
    \item Russell, S., \& Norvig, P. (2020). \textit{Artificial Intelligence: A Modern Approach}. Pearson.
    \item Documentación oficial de Python 3: \texttt{https://docs.python.org/3/}
    \item Giarratano, J., \& Riley, G. (2004). \textit{Expert Systems: Principles and Programming}. PWS Publishing.
\end{itemize}

\end{document}
